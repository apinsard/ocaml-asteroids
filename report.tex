\documentclass[a4paper,10pt,openany,oneside]{report}

\usepackage[utf8]{inputenc}
\usepackage[T1]{fontenc}
\usepackage[francais]{babel}
\usepackage{eurosym}
\usepackage[margin=1.5in]{geometry}
\usepackage{graphicx}
\usepackage{fancyhdr}
\usepackage{xcolor}

\pagestyle{fancy}
\renewcommand{\headrulewidth}{0pt}

\setlength{\oddsidemargin}{0pt}
\setlength{\topmargin}{0pt}
\setlength{\marginparwidth}{0pt}
\setlength{\headheight}{30pt}
\setlength{\headwidth}{450pt}
\setlength{\headsep}{10pt}
\setlength{\voffset}{0pt}
\setlength{\hoffset}{0pt}
\setlength{\footskip}{70pt}
\setlength{\textwidth}{450pt}
\setlength{\textheight}{580pt}
\setlength{\parskip}{8pt}
\setlength{\parindent}{0pt}

\title{Asteroids - Projet PRG2}
\author{Théo \bsc{Nazé} et Antoine \bsc{Pinsard}}
\date{5 Avril 2015}

\makeatletter
\lhead{\Large{\bf\@title}}
\lfoot{\footnotesize \color[gray]{0.5} \@title}
\rfoot{\footnotesize \color[gray]{0.5} \@author / \@date}
\makeatother

\renewcommand{\thesection}{\arabic{section}}

\begin{document}

\maketitle

\section{Introduction}

La majeur partie du projet réside dans la gestion des formes et déplacements
dans le plan.

Afin de gérer cela, nous avons décidé d'attribuer à chaque objet (vaisseau,
astéroide, missile), une position (celle de son centre de gravité), une
orientation et une vitesse. Ainsi, à chaque étape, nous modifions la position
du centre de gravité en fonction de la vitesse et de l'orientation. Puis nous
recalculons la position des différents sommets du polygone en fonction de la
taille de l'objet, de son centre de gravité et de son orientation.

\end{document}
