\documentclass[a4paper,10pt,openany,oneside]{report}

\usepackage[utf8]{inputenc}
\usepackage[T1]{fontenc}
\usepackage[francais]{babel}
\usepackage{eurosym}
\usepackage[margin=1.5in]{geometry}
\usepackage{graphicx}
\usepackage{fancyhdr}
\usepackage{xcolor}
\usepackage{mathtools}

\pagestyle{fancy}
\renewcommand{\headrulewidth}{0pt}

\setlength{\oddsidemargin}{0pt}
\setlength{\topmargin}{0pt}
\setlength{\marginparwidth}{0pt}
\setlength{\headheight}{30pt}
\setlength{\headwidth}{450pt}
\setlength{\headsep}{10pt}
\setlength{\voffset}{0pt}
\setlength{\hoffset}{0pt}
\setlength{\footskip}{70pt}
\setlength{\textwidth}{450pt}
\setlength{\textheight}{580pt}
\setlength{\parskip}{8pt}
\setlength{\parindent}{0pt}

\title{Asteroids - Projet PRG2}
\author{Théo \bsc{Nazé} et Antoine \bsc{Pinsard}}
\date{5 Avril 2015}

\makeatletter
\lhead{\Large{\bf\@title}}
\lfoot{\footnotesize \color[gray]{0.5} \@title}
\rfoot{\footnotesize \color[gray]{0.5} \@author / \@date}
\makeatother

\renewcommand{\thesection}{\arabic{section}}

\begin{document}

\maketitle

\section{Introduction}

La majeur partie du projet réside dans la gestion des déplacements des formes
dans le plan.

Afin de gérer cela, nous avons décidé d'attribuer à chaque objet (vaisseau,
astéroide, missile), une position (celle de son centre de gravité), une
orientation et une vitesse. Ainsi, à chaque étape, nous modifions la position
du centre de gravité en fonction de la vitesse et de l'orientation. Puis nous
recalculons la position des différents sommets du polygone en fonction de la
taille de l'objet, de son centre de gravité et de son orientation. Pour la
rotation du vaisseau, il suffit de modifier son orientation.

Se pose ensuite le problème des collisions. Nous avons d'abord pensé à calculer
l'intersection entre chaque astéroide/missile et entre le vaisseau et chaque
astéroide. Afin de simplifier le problème nous avons finalement décider de
calculer la distance entre les deux éléments. Si la distance est inférieure à
la somme des rayons des objets, on peut considère une collision. Cette méthode
est moins précise que celle de l'intersection, mais plus simple à mettre en
place.

\section{Analyse du problème}

Le calcule de l'état suivant se découpe en 5 étapes :

\begin{itemize}
  \item Calcul du nouvel état du vaisseau
  \item Calcul du nouvel état des missiles
  \item Calcul du nouvel état des astéroides
  \item Vérification des collisions asteroide/vaisseau
  \item Vérification des collisions astéroide/missile et recalcul de l'état des
    astéroides le cas échéant.
\end{itemize}

\subsection{Calcul du nouvel état du vaisseau}

L'accélération n'étant pas implémentée, et la rotation étant traitée
directement à l'enfoncement des touche "j" et "l", cette étape est omise.

\subsection{Calcul du nouvel état des missiles}

Le déplacement d'un missile se fait à l'aide de la formule suivante :

\[new\_pos = pos + (\cos (orientation) * vitesse, \sin(orientation) * vitesse)\]

Si la nouvelle position sort du cadre de jeu, le missile est supprimé.

\subsection{Calcul du nouvel état des astéroides}

Le calcul du déplacement d'un astéroide se fait de la même manière que pour
celui d'un missile :

\[new\_pos = pos + (\cos (orientation) * vitesse, \sin(orientation) * vitesse)\]

En revanche, la position est calculée modulo (largeur du cadre, hauteur du
cadre) de manière à faire réapparaitre l'astéroide de l'autre côté de l'écran
lorsqu'il sort d'un côté.

\subsection{Vérification des collisions astéroide/vaisseau}

On parcourt la liste des astéroides et 

\subsection{Vérification des collisions astéroide/missile}

\section{Bilan}

\textit{(* À compléter *)}

\section{Exécuter le programme}

\texttt{./asteroids}

\section{Rapport de test}

\textit{(* À compléter *)}

\section{Conclusion}

La réalisation du projet nous a pris deux jours. Les difficultées que nous
avons rencontrés sont le manque de documentation du langage. Très peu de site
référencent les bibliothèques et built-ins d'Ocaml. Globalement, la seule
ressource valable est \texttt{caml.inria.fr}. Il est aussi assez difficile
d'exploiter l'interpréteur interactif pour effectuer de petits tests et
vérifier rapidement des résultats attendus. En effet il n'est pas possible
d'utiliser les flèches directionnelles pour se déplacer sur la ligne courante
et rappeler des commandes précédemment exécutées.

\end{document}
