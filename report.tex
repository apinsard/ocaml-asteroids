\documentclass[a4paper,10pt,openany,oneside]{report}

\usepackage[utf8]{inputenc}
\usepackage[T1]{fontenc}
\usepackage[francais]{babel}
\usepackage{eurosym}
\usepackage[margin=1.5in]{geometry}
\usepackage{graphicx}
\usepackage{fancyhdr}
\usepackage{xcolor}

\pagestyle{fancy}
\renewcommand{\headrulewidth}{0pt}

\setlength{\oddsidemargin}{0pt}
\setlength{\topmargin}{0pt}
\setlength{\marginparwidth}{0pt}
\setlength{\headheight}{30pt}
\setlength{\headwidth}{450pt}
\setlength{\headsep}{10pt}
\setlength{\voffset}{0pt}
\setlength{\hoffset}{0pt}
\setlength{\footskip}{70pt}
\setlength{\textwidth}{450pt}
\setlength{\textheight}{580pt}
\setlength{\parskip}{8pt}
\setlength{\parindent}{0pt}

\title{Asteroids - Projet PRG2}
\author{Théo \bsc{Nazé} et Antoine \bsc{Pinsard}}
\date{5 Avril 2015}

\makeatletter
\lhead{\Large{\bf\@title}}
\lfoot{\footnotesize \color[gray]{0.5} \@title}
\rfoot{\footnotesize \color[gray]{0.5} \@author / \@date}
\makeatother

\renewcommand{\thesection}{\arabic{section}}

\begin{document}

\maketitle

\section{Introduction}

La majeur partie du projet réside dans la gestion des formes et déplacements
dans le plan.

Afin de gérer cela, nous avons décidé d'attribuer à chaque objet (vaisseau,
astéroide, missile), une position (celle de son centre de gravité), une
orientation et une vitesse. Ainsi, à chaque étape, nous modifions la position
du centre de gravité en fonction de la vitesse et de l'orientation. Puis nous
recalculons la position des différents sommets du polygone en fonction de la
taille de l'objet, de son centre de gravité et de son orientation. Pour la
rotation du vaisseau, il suffit de modifier son orientation.

Se pose ensuite le problème des collisions. Pour cela, après avoir calculé
l'état suivant, il faut vérifier l'intersection entre le vaisseau et chaque
astéroide. Si elle est vide, on ne fait rien. Si elle n'est pas vide, la partie
est perdue. Il faut également vérifier l'intersection entre chaque astéroide et
chaque missile. Si l'intersection est vide, on ne fait rien. Si elle n'est pas
vide, on supprime le missile et lance le comportement d'explosion de
l'astéroide.

\section{Analyse du problème}

Le calcule de l'état suivant se découpe en 5 étapes :

\begin{itemize}
  \item Calcul du nouvel état du vaisseau
  \item Calcul du nouvel état des missiles
  \item Calcul du nouvel état des astéroides
  \item Vérification des collisions asteroide/vaisseau
  \item Vérification des collisions astéroide/missile et recalcul de l'état des
    astéroides le cas échéant.
\end{itemize}

\section{Bilan}

\textit{(* À compléter *)}

\section{Exécuter le programme}

\texttt{./asteroids}

\section{Rapport de test}

\textit{(* À compléter *)}

\section{Conclusion}

La réalisation du projet nous a pris deux jours. Les difficultées que nous
avons rencontrés sont le manque de documentation du langage. Très peu de site
référencent les bibliothèques et built-ins d'Ocaml. Globalement, la seule
ressource valable est \texttt{caml.inria.fr}. Il est aussi assez difficile
d'exploiter l'interpréteur interactif pour effectuer de petits tests et
vérifier rapidement des résultats attendus. En effet il n'est pas possible
d'utiliser les flèches directionnelles pour se déplacer sur la ligne courante
et rappeler des commandes précédemment exécutées.

\end{document}
